\section{Introduction}

Time to Recache(TTR) is an interesting metric to measure cache performance, but thus far few case studies exist to show its usefulness~\cite{spjutpugsley11,carterkorbel13}. This paper represents one such case study to show how TTR can be used to assist a cache designer in finding the optimal cache configuration. While in this particular example, it would have been just as practical to use another cache metric, like hit rate, or misses per 1000 instructions, TTR was useful in this case.

The idea behind this is that if you were to run a program with a given cache configuration and replacement strategy, you would want to minimize the miss rate. 
However if you only have the miss rate, you have to blindly guess what cache configuration would improve the miss rate. 
This means that you essentially have to test all possibilities in the solution space, which can take a significant time to simulate. 
By using TTR as a guide, you can gain insight into what will improve your miss rate, which allows you to selectively examine cache configurations to get to the optimal solution.
This could potentially save an engineer considerable time waiting for simulations to run in situations where computing resources are limited or a quick decision is required.




